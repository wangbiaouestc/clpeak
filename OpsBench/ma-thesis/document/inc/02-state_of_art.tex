\chapter{Stand der Technik} 
Die folgenden Abschnitte sind aus \cite{wikipedia-latex} entnommen:

\section{Geschichte}

Das Programm \TeX{} wurde von Donald E. Knuth, Professor an der
Stanford-University, entwickelt \cite{Knuth84}. Leslie Lamport entwickelte
Anfang der 1980er Jahre \cite{Lamport94} darauf aufbauend \LaTeX{}, eine
Sammlung von \TeX{}-Makros. Der Name ist eine Abkürzung für Lamport \TeX{}.
Heute ist \LaTeX{} die populärste Methode, \TeX{} zu verwenden.

\section{Grundprinzip}

Im folgenden wird eine Auswahl an Grundprinzipien von \LaTeX{} vorgestellt.

\subsection{Kein WYSIWYG}

Im Gegensatz zu anderen Textverarbeitungen, die nach dem
\emph{What-you-see-is-what-you-get-System} arbeiten, arbeitet der Autor mit
Textdateien, in denen er innerhalb eines Textes anders zu formatierende
Passagen oder Überschriften mit Befehlen textuell auszeichnet. Das Beispiel
unten zeigt den Quellcode eines einfachen \LaTeX{}-Dokuments. Bevor das
\LaTeX{}"=System den Text ansprechend setzen kann, muss es den Quellcode erst
verarbeiten. Das dabei von \LaTeX{} generierte Layout gilt als sehr sauber,
sein Formelsatz als sehr ausgereift. Außerdem ist die Ausgabe u. a. nach PDF,
HTML und PostScript möglich. \LaTeX{} eignet sich insbesondere für umfangreiche
Arbeiten wie Diplomarbeiten und Dissertationen, die oftmals strengen
typographischen Ansprüchen genügen müssen. Insbesondere in der Mathematik und
den Naturwissenschaften erleichtert \LaTeX{} das Anfertigen von Schriftstücken
durch seine komfortablen Möglichkeiten der Formelsetzung gegenüber
herkömmlichen Textverarbeitungen. Das Verfahren von \LaTeX{} wird auch gerne
mit WYGIWYM (\emph{what you get is what you mean}) umschrieben.

Das schrittweise Arbeiten erfordert vordergründig im Vergleich zu herkömmlichen
Textverarbeitungen einerseits eine längere Einarbeitungszeit, andererseits kann
das Aussehen des Resultats genau festgelegt werden. Die längere
Einarbeitungszeit kann sich jedoch, insbesondere bei Folgeprojekten mit
vergleichbarem Umfang oder ähnlichen Erfordernissen, lohnen \cite{Fenn09}.
Inzwischen gibt es auch grafische Editoren, die mit \LaTeX{} arbeiten können
und WYSIWYG oder WYGIWYM bieten. Beispiele hierfür sind LyX und BaKoMa \TeX{},
welche ungeübten Usern den Einstieg deutlich erleichtern können.


\subsection{Logisches Markup}

Bei der Benutzung von \LaTeX{} wird ein sogenanntes \emph{logisches Markup}
verwendet.  Soll beispielsweise in einem Dokument eine Überschrift erstellt
werden, so wird der Text nicht wie in \TeX{} rein optisch hervorgehoben (etwa
durch Fettdruck mit größerer Schrift, also: 
\verb.\font\meinfont=cmb10 at 24pt \meinfont Einleitung.), sondern die
Überschrift wird als solche gekennzeichnet (z. B. mittels
\verb=\section{Einleitung}=). In den Klassen- oder \emph{sty}"=Dateien wird
global festgelegt, wie eine derartige Abschnittsüberschrift zu gestalten ist:
\glqq{}das Ganze fett setzen; mit einer Nummer davor, die hochzuzählen ist; den
Eintrag in das Inhaltsverzeichnis vorbereiten\grqq{} usw. Dadurch erhalten alle
diese Textstellen eine einheitliche Formatierung. Außerdem wird es dadurch
möglich, automatisch aus allen Überschriften im Dokument mit dem Befehl
\verb=\tableofcontents= ein Inhaltsverzeichnis zu generieren.

\subsection{Rechnerunabhängigkeit}

Wie \TeX{} selbst ist \LaTeX{} unabhängig von Hardware und Betriebssystemen
benutzbar. Mehr noch, die Ausgabe (Zeilen- und Seitenumbrüche) ist genau
gleich, unabhängig von der verwendeten Rechnerplattform und dem verwendeten
Drucker – wenn alle verwendeten Zusatzpakete (siehe unten) in geeigneten
Versionen installiert sind. \LaTeX{} ist auch nicht auf die Schriftarten des
jeweiligen Betriebssystems angewiesen, die oftmals für die Anzeige am
Bildschirm und nicht für den Druck ausgelegt sind, sondern enthält eine Reihe
von eigenen Schriftarten.

\subsection{Verbreitung}

Aufgrund seiner Stabilität, der freien Verfügbarkeit für viele Betriebssysteme
und dem ausgezeichneten Formelsatz sowie seinen Features speziell für
wissenschaftliche Arbeiten wird \LaTeX{} vor allem an Universitäten und
Hochschulen benutzt; insbesondere in der Mathematik und den Naturwissenschaften
ist \LaTeX{} die Standardanwendung für wissenschaftliche Arbeiten
\cite{ctan-friends,Gaudeul06}. Es gibt auch spezielle Pakete für andere
Fachbereiche, etwa zum Notensatz für Musiker, zur Ausgabe von Lautschrift für
Linguisten, zum Setzen von altsprachlichen Texten für Altphilologen oder zum
Bibliografieren für Juristen und Geisteswissenschaftler. Auch einige
Unternehmen setzen \LaTeX{} ein, unter anderem um Handbücher, Fahrpläne und
Produktkataloge zu erzeugen.
